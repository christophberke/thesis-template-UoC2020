\chapter{Inner layouting, formatting and style}
\section{Formatting math environments and equations}

\subsection{Multi-line equations}
Use \verb|align| for equations that extend over several lines. Do \textbf{not} use \verb|eqnarray|.

\subsection{The golden rules: Interplay of equations and text}
When it comes to embedding equations in continuous text, I can only recommend David Mermin's epoch-making text \href{https://wp.optics.arizona.edu/kupinski/wp-content/uploads/sites/91/2023/05/MerminEquations.pdf}{`What's wrong with these equations?'} to everyone. Let me quote his three golden rules directly from the article (Highlighting in bold by me):
\paragraph{Rule 1 (Fisher's rule)} \textbf{Number all displayed equations.} The most common violation of Fisher's rule is the misguided practice of numbering only those displayed equations to which the text subsequently refers back.
\paragraph{Rule 2 (Good Samaritan rule)} When referring to an equation identify it by a phrase as well as a number. No compassionate and helpful person would herald the arrival of Eq.~(7.38) by saying ``inserting (2.47) and (3.51) into (5.13)'' when it is possible to say ``inserting the form (2.47) of the electric field \textbf{E} and the Lindhard form (3.51)\dots''. [...] Consistent use of the Good Samaritan rule might well increase the lenght of your paper by a few percent. But admit it. Your paper is already to long by at least 30\% because you were in such a rush to get it out.
\paragraph{Rule 3 (Math is prose rule)} \textbf{End a displayed equation with a punctuation mark.} It is implicit in this statement that the absence of a punctuation mark is itself a degenerate form of punctuation that, like periods, commas or semicolons, can be used \emph{provided it makes sense.}

There is not much to add. Consistently following these rules makes the text much more readable and easier to discuss. As a tiny elaboration on the last point, compare the following two equations:
\begin{align}
	|\langle \phi (t) | \psi(t) \rangle| = |\langle \phi(0) | \op{U}^{-1}(t) \op{U}(t) | \psi(0)  \rangle| = |\langle \phi (0) | \psi(0) \rangle|. \\ 
	|\langle \phi (t) | \psi(t) \rangle| = |\langle \phi(0) | \op{U}^{-1}(t) \op{U}(t) | \psi(0)  \rangle| = |\langle \phi (0) | \psi(0) \rangle|\,.
\end{align}
The difference is hard to spot, in fact the distance between the end of the equation and the following punctuation mark is slightly larger in the second variant, where I inserted \verb|\,| before the punctuation mark.
This may seem completely exaggerated, and it probably is, but at least the people in the Trebst group can be told that the use of the lower variant will be noticed favorably\ldots
Concerning the first rule, note that a multi-line equations generally needs only a single number, i.e.,
\begin{align}
	O(t) &= \langle \psi(t) | \op{O} | \psi(t) \rangle = \sum_{i,j} c_i^* c_j e^{i \left( E_i - E_j \right)t} \langle i |\op{O}| j\rangle  \nonumber \\ & = \sum_i |c_i|^2 \langle i |\op{O}| i \rangle + \sum_{i,j\neq i} c_i^*c_j e^{i \left( E_i - E_j \right) t} \langle i |\op{O}| j\rangle\,,
\end{align}
but
\begin{align} 
	\mathrm{IPR} &= \sum_\alpha |c_i^\alpha|^4\,,\\
	S_P &= - \sum_\alpha |c_i^\alpha|^2 \ln |c_i^\alpha|^2 \,.
\end{align}
If you have a single equation that extends over multiple lines and you want to number each line individually, like \eqref{eq:tauZZa} and \eqref{eq:tauZZb}, the \verb|subequations| environment might be a good choice, yielding
\begin{subequations}
\begin{align}
	\op{H}_\text{MBL} &= E_0 + \sum \limits_i h_i \op{\tau}_i^z + \sum \limits_{i>j} J_{ij} \op{\tau}_i^z \op{\tau}_j^z + \sum \limits_{i>j>k} J_{ijk} \op{\tau}_i^z \op{\tau}_j^z \op{\tau}_k^z + \dots \label{eq:tauZZa} \\&= \sum_\vect{b} c_\vect{b} \op{Z}_1^{b_1} \op{Z}_2^{b_2} \dots \op{Z}_L^{b_L}\,. \label{eq:tauZZb}
\end{align}
\end{subequations}

\subsection{Indices}
Any index consisting of more than one letter should not be italicized.
If an index only consists of a single letter, then it is okay not to type it as, e.g., \verb|$E_k$|, yielding $E_k$. But $E_{kin}$ looks horrible! Instead, write \verb|$E_\text{kin}$|, yielding $E_\text{kin}$.
Writing $cos(x)$ (\verb|$cos(x)$|) instead of $\cos(x)$ (\verb|\cos(x)|) falls into the same sacrilegious category.

\subsection{Displaying huge numbers}
In both, American English and British English, one uses a comma to separate groups of thousands, for example $3{,}125{,}500$. One can enclose the comma in braces, \verb|3{,}125{,}500|, to get an optimal and stable spacing between the numbers. 


displaying huge numbers.
\section{Tables}
\subsection{Booktabs}


\section{How to make nice tables?}
Most importantly, do not use vertical rules!
From `The Chicago Manual of Style' \cite{chicagoMOS}: ``To produce a clear, professional-looking table, rules should be used sparingly. Many tables will require just three rules, all of them horizontal—one at the very top of the table, below the title and above the column heads; one just below the column heads; and one at the bottom of the table, along the bottom of the last row, above any notes to the table. (\ldots) Vertical rules should be used sparingly (\ldots).'' 
Use the \verb|booktabs| package, see  \href{https://ctan.org/pkg/booktabs}{here} for the documentation and \href{https://nhigham.com/2019/11/19/better-latex-tables-with-booktabs/}{here} or \href{}{} for some examples and a discussion why \verb|booktabs| is the way to go. Remember that table captions usually go above the table. See \tabref{tab:table1} for an example with multiple hierachy levels in $x$ and $y$ direction and \tabref{tab:table2} for an example of a side-by-side table with colored rows.

\begin{table}
	
	\centering
	\caption{\textbf{Example for table with different hierachy levels in $x$ and $y$ direction.} \blindtext}
	\label{tab:table1}
	\vspace{5ex}
	\begin{tabular}{@{}rrrrcrrr@{}}\toprule
		& \multicolumn{3}{c}{$w = 8$} & \phantom{abc}& \multicolumn{3}{c}{$w = 16$} \\
		\cmidrule{2-4} \cmidrule{6-8}
		& $t=0$ & $t=1$ & $t=2$ && $t=0$ & $t=1$ & $t=2$\\ 
		\midrule
		$\mathrm{dir}=1$\\
		$c$ & 0.0790 & 0.1692 & 0.2945 && 0.3670 & 0.7187 & 3.1815 \\
		$c$ & -0.8651& 50.0476& 5.9384&& -9.0714& 297.0923& 46.2143\\
		$c$ & 124.2756& -50.9612& -14.2721&& 128.2265& -630.5455& -381.0930\\
		$\mathrm{dir}=0$\\
		$c$ & 0.0357& 1.2473& 0.2119&& 0.3593& -0.2755& 2.1764\\
		$c$ & -17.9048& -37.1111& 8.8591&& -30.7381& -9.5952& -3.0000\\
		$c$ & 105.5518& 232.1160& -94.7351&& 100.2497& 141.2778& -259.7326\\
		\bottomrule
	\end{tabular}

\end{table}

\begin{table}
	\centering 
	\caption{\textbf{Example of side-by-side table and colored rows.} \blindtext}
	\label{tab:table2}
	\vspace{5ex}
	\begin{tabular}{ccccrr} 
		\toprule
		$l_1$ & $l_2$ & $l_3$ & $l_4$ & $\Delta N_\text{ex}$ & $|\langle \psi | \op{H}_\text{int} | \phi \rangle|$  \\ 
		\midrule 
		\rowcolor{pqred} 0 & 1 & 3 & 0 & 0 & ---\\
		0 & 0 & 0 & 0 & $-4$ & 0.04\\
		0 & 0 & 2 & 0 & $-2$ & 1.88\\
		\rowcolor{pqblue} 0 & 0 & 4 & 0 & 0 & 2.06\\
		0 & 0 & 6 & 0 & 2 & 0.32\\
		0 & 0 & 8 & 0 & 4 & 0.09\\
		0 & 1 & 0 & 1 & $-2$ & 0.04\\
		\rowcolor{pqyellow} 0 & 1 & 0 & 3 & 0 & 0.002\\
		0 & 1 & 0 & 5 & 2 & 0.0001\\
		0 & 1 & 0 & 7 & 4 & 0.00002\\
		\rowcolor{pqblue} 0 & 1 & 2 & 1 & 0 & 1.89\\
		\bottomrule
	\end{tabular}
	\hspace{0.5cm}
	\begin{tabular}{ccccrr} 
		\toprule
		$l_1$ & $l_2$ & $l_3$ & $l_4$ & $\Delta N_\text{ex}$ & $|\langle \psi | \op{H}_\text{int} | \phi \rangle|$  \\ 
		\midrule 
		0 & 2 & 0 & 0 & $-2$ & 0.06\\
		\rowcolor{pqblue} 0 & 2 & 2 & 0 & 0 & 2.57\\
		0 & 2 & 4 & 0 & 2 & 2.82\\
		0 & 2 & 6 & 0 & 4 & 0.44\\
		\rowcolor{pqyellow} 0 & 4 & 0 & 0 & 0 & 0.003\\
		0 & 4 & 2 & 0 & 2 & 0.15\\
		0 & 4 & 4 & 0 & 4 & 0.17\\
		0 & 6 & 0 & 0 & 2 & 0.0005\\
		0 & 6 & 2 & 0 & 4 & 0.02\\
		0 & 8 & 0 & 0 & 4 & 0.0001\\
		\rowcolor{pqblue} 1 & 0 & 3 & 0 & 0 & 1.17\\
		\bottomrule
	\end{tabular}
\end{table}





\subsection{Colored rows / cells}
\section{Figures}
\subsection{Side by side}
\subsection{Hyperlinks to subpanels}
\section{Fine-tuning}
Immer wenn ich einen der folgenden Fehler sehe, dann baut sich bei mir direkt der Gedanke auf, dass der Autor jetzt direkt in der Bringschuld ist, mir zu beweisen, dass er kein Idiot ist. Mit anderen Worten: Wenn ich so etwas sehe kriege ich schlechte Laune und bin dem Inhalte gegenüber nicht mehr neutral (sogar mehr noch, als wenn ich inhaltliche Fehler sehe). Klingt übertrieben? Mag sein, aber wenn man seine Leserschaft nicht kennt, sollte man sich trotzdem Mühe geben, es direkt richtig zu machen. Wer weiß an wen man gerät.
\subsection{Quotation marks}
Für wörtliche Zitationen gibt es keine einheitlichen Regeln, ob man diese durch einzelne oder doppelte Anführungszeichen kennzeichnet, also so: `Dies ist ein wörtlich übernommenes Zitat' oder so ``Dies ist ein wörtlich übernommenes Zitat''. Wie immer gilt: Entscheidung treffen und dann konsistent in der ganzen Thesis bei dieser Entscheidung bleiben. Ganz wichtig: Anführungszeichen in LaTeX immer so \verb|`...'| (einzeln) oder so \verb|``...''| typesetten. Wenig sieht schlimmer aus als ''ein falsch gesetztes Anführungszeichen''.

\subsection{Hyphens and dashes}

There a three lengths of hyphens, the regular hyphen \verb|-|, the en dash \verb|--| and the em dash \verb|---|.
This is what they look like when typeset: - vs. -- vs. ---.
In my thesis, I used them as recommended in the Chicago manual of style:
\begin{itemize}
\item The hyphen (\verb|-|) is used to connect words that function together as a single concept or work together as a joint modifier, i.e., `well-defined concept' or `collision-free device'. Never use the single hyphen similar to brackets, e.g., to separate thoughts or similar.
\item The en dash (\verb|--|) connects things that are related by some form of distance, e.g. `pages 12--17', the `May--September issue', the `cycling race Milan--San Remo' or `a coupling strength of 10--15 MHz' (although for the latter `10 to 15 MHz' is probably better). A rare use case for typesetting nerds: Use \verb|--| instead of \verb|-| in a compound adjective. Specifically, an en dash is preferred when one element of the compound is itself an open compound. For example, the prefix post- is usually connected to the following word with a hyphen, but to connect it to the compound noun World War II, it's better to use an en dash, i.e., post--World War II vs. post-World War II.
\item The em dash should be used to separate additional thoughts---like this. Here is an example from my thesis: `The insights gained for the simpler model---in particular, the existence of a quantum chaotic region---retain their validity for more elaborate frequency arrangements.' I use the em dashs without spaces as is usually recommended. Some style guides recommend the use of spaces --- like this --- and one also encounters the combination of en dashes with spaces -- like this. As always, the most important thing is consistency. Decide on a variant and use it consistently everywhere and always.
\end{itemize}



\section{Some stylistic advices}
Below is a collection of stilistic questions I have thought about for too long.
\subsection{Hyphen vs. en dash vs. em dash}

\subsection{Hurenkinder [sic!]}
\LaTeX gibt sein bestes um ein stimmiges Erscheinungsbild hinzubekommen. Dazu gehört unter Anderem, dass versucht wird zu vermeiden, dass einzelne Zeilen auf einer neuen Seite stehen. Man stelle sich z.B. die Situation vor, dass ein Kapitel mit einer einzelnen Zeile auf einer ungeraden Seite anfängt. Zum einen steht diese Zeile dann recht verloren da, was ästhetisch sehr unbefriedigend ist. Außerdem verlängert das die ganze Arbeit direkt um zwei Seiten, da jedes neue Kapitel auf einer neuen ungeraden Seite beginnt. So etwas sollte also nach Moglichkeit vermieden werden. Man nennt das im englischen ubrigens Orphan (`the orphan does not know where he is coming from'). Im Deutschen gibt es den sehr viel lustigeren Namen `Hurenkind' (again, the Hurenkind does not know where he is coming from). 
Ebenfalls unschoen, aber nicht ganz so schlimm ist, wenn ein neuer Absatz oder eine neue Section ganz unten auf einer Seite beginnt, so dass nur noch ein oder zwei Zeilen vor dem Seitenumbruch dort stehen. Diese nennt man im Englischen ... und im Deutschen ... 
Wie gesagt versucht LaTex so etwas zu vermeiden. Man kann LaTeX dazu bringen, dass noch mehr zu vermeinden, in dem man die penalties fuer das Auftreten von Orphans und Widows nach oben setzt. Diese geschieht ueber
die Befehle
xxx
xxx
Es ist besser diese Befehle zu benutzen als (im Falle von Orphans) manuell einen Zeilenumbruch zu erzwingen, weil es leicht passieren kann, dass Aenderungen in vorherigen Abschnitten das Problem automatisch loesen und man dann nicht das Risiko einer halb leeren Seite o.ä. eingeht.


\subsection{Nonlinear vs. non-linear and related issues}
The prefix `non' appears very often and with many different following words. My impression was that it is not treated uniformly in the literature (even by the same authors) and that one finds both variants. In fact, in American English (AE), the hyphen is mostly omitted, whereas in British English (BE) it is commonly used. Compare e.g., the two reviews \cite{abaninRecentProgressManybody2017} and \cite{abaninColloquiumManybodyLocalization2019}. Ref.~\cite{abaninRecentProgressManybody2017} was published by Annals of Physics (Berlin) which belongs to the german Wiley-VCH publisher and uses BE. Therefore one finds \emph{non-quilibrium,non-entangled,non-uniform,non-thermal,non-ergodic,non-local,non-trivial,non-interacting,non-Abelian,} etc.
Ref.~\cite{abaninColloquiumManybodyLocalization2019} was published in Reviews of modern physics, an APS journal. In consequence, they write \emph{nonequilibrium,nonzero,noninteracting,nonuniform,nonthermal,nonvanishing,nonlocal,nonthermalizing}, but however, they write \emph{non-Abelian}. I went for the AE option, because I used AE everywhere else (behavior, neighbor, gray, etc.).
When browsing papers to find out how issues like this are handled, the best idea is not to take the arXiv versions, as there is often no consistency even within a single paper, but the published versions that have undergone some form of post-editing by the publisher.
