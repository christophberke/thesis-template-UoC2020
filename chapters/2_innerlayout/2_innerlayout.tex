\chapter{Inner layouting, formatting and style}
\section{Formatting math environments and equations}

\subsection{Multi-line equations}
Use \verb|align| for equations that extend over several lines. Do \textbf{not} use \verb|eqnarray|.

\subsection{The golden rules: Interplay of equations and text}
When it comes to embedding equations in continuous text, I can only recommend David Mermin's epoch-making text \href{https://wp.optics.arizona.edu/kupinski/wp-content/uploads/sites/91/2023/05/MerminEquations.pdf}{`What's wrong with these equations?'} to everyone. Let me quote his three golden rules directly from the article (Highlighting in bold by me):
\paragraph{Rule 1 (Fisher's rule)} \textbf{Number all displayed equations.} The most common violation of Fisher's rule is the misguided practice of numbering only those displayed equations to which the text subsequently refers back.
\paragraph{Rule 2 (Good Samaritan rule)} When referring to an equation identify it by a phrase as well as a number. No compassionate and helpful person would herald the arrival of Eq.~(7.38) by saying ``inserting (2.47) and (3.51) into (5.13)'' when it is possible to say ``inserting the form (2.47) of the electric field \textbf{E} and the Lindhard form (3.51)\dots''. [...] Consistent use of the Good Samaritan rule might well increase the lenght of your paper by a few percent. But admit it. Your paper is already to long by at least 30\% because you were in such a rush to get it out.
\paragraph{Rule 3 (Math is prose rule)} \textbf{End a displayed equation with a punctuation mark.} It is implicit in this statement that the absence of a punctuation mark is itself a degenerate form of punctuation that, like periods, commas or semicolons, can be used \emph{provided it makes sense.}

There is not much to add. Consistently following these rules makes the text much more readable and easier to discuss. As a tiny elaboration on the last point, compare the following two equations:
\begin{align}
	|\langle \phi (t) | \psi(t) \rangle| = |\langle \phi(0) | \op{U}^{-1}(t) \op{U}(t) | \psi(0)  \rangle| = |\langle \phi (0) | \psi(0) \rangle|. \\ 
	|\langle \phi (t) | \psi(t) \rangle| = |\langle \phi(0) | \op{U}^{-1}(t) \op{U}(t) | \psi(0)  \rangle| = |\langle \phi (0) | \psi(0) \rangle|\,.
\end{align}
The difference is hard to spot, in fact the distance between the end of the equation and the following punctuation mark is slightly larger in the second variant, where I inserted \verb|\,| before the punctuation mark.
This may seem completely exaggerated, and it probably is, but at least the people in the Trebst group can be told that the use of the lower variant will be noticed favorably\ldots
Concerning the first rule, note that a multi-line equations generally needs only a single number, i.e.,
\begin{align}
	O(t) &= \langle \psi(t) | \op{O} | \psi(t) \rangle = \sum_{i,j} c_i^* c_j e^{i \left( E_i - E_j \right)t} \langle i |\op{O}| j\rangle  \nonumber \\ & = \sum_i |c_i|^2 \langle i |\op{O}| i \rangle + \sum_{i,j\neq i} c_i^*c_j e^{i \left( E_i - E_j \right) t} \langle i |\op{O}| j\rangle\,,
\end{align}
but
\begin{align} 
	\mathrm{IPR} &= \sum_\alpha |c_i^\alpha|^4\,,\\
	S_P &= - \sum_\alpha |c_i^\alpha|^2 \ln |c_i^\alpha|^2 \,.
\end{align}
If you have a single equation that extends over multiple lines and you want to number each line individually, like \eqref{eq:tauZZa} and \eqref{eq:tauZZb}, the \verb|subequations| environment might be a good choice, yielding
\begin{subequations}
\begin{align}
	\op{H}_\text{MBL} &= E_0 + \sum \limits_i h_i \op{\tau}_i^z + \sum \limits_{i>j} J_{ij} \op{\tau}_i^z \op{\tau}_j^z + \sum \limits_{i>j>k} J_{ijk} \op{\tau}_i^z \op{\tau}_j^z \op{\tau}_k^z + \dots \label{eq:tauZZa} \\&= \sum_\vect{b} c_\vect{b} \op{Z}_1^{b_1} \op{Z}_2^{b_2} \dots \op{Z}_L^{b_L}\,. \label{eq:tauZZb}
\end{align}
\end{subequations}

\subsection{Indices}
Any index consisting of more than one letter should not be italicized.
If an index only consists of a single letter, then it is okay not to type it as, e.g., \verb|$E_k$|, yielding $E_k$. But $E_{kin}$ looks horrible! Instead, write \verb|$E_\text{kin}$|, yielding $E_\text{kin}$.
Writing $cos(x)$ (\verb|$cos(x)$|) instead of $\cos(x)$ (\verb|\cos(x)|) falls into the same sacrilegious category.

\subsection{Displaying huge numbers}
In both, American English and British English, one uses a comma to separate groups of thousands, for example $3{,}125{,}500$. One can enclose the comma in braces, \verb|3{,}125{,}500|, to get an optimal and stable spacing between the numbers. 


displaying huge numbers.
\section{Tables}
\subsection{Booktabs}
\subsection{Colored rows / cells}
\section{Figures}
\subsection{Side by side}
\subsection{Hyperlinks to subpanels}
\section{Fine-tuning}
Immer wenn ich einen der folgenden Fehler sehe, dann baut sich bei mir direkt der Gedanke auf, dass der Autor jetzt direkt in der Bringschuld ist, mir zu beweisen, dass er kein Idiot ist. Mit anderen Worten: Wenn ich so etwas sehe kriege ich schlechte Laune und bin dem Inhalte gegenüber nicht mehr neutral (sogar mehr noch, als wenn ich inhaltliche Fehler sehe). Klingt übertrieben? Mag sein, aber wenn man seine Leserschaft nicht kennt, sollte man sich trotzdem Mühe geben, es direkt richtig zu machen. Wer weiß an wen man gerät.
\subsection{Quotation marks}
\subsection{Hyphens}
\subsection{nonlinear vs. non-linear}
\subsection{Orphans and widows}