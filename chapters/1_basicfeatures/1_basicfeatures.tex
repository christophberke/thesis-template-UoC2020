\graphicspath{{chapters/1_basicfeatures/figures/}}

\chapter{Basic features of the template}
To compile the thesis, generate the bibliography, and ensure flawless cross-references between the table of contents, main text, citations, and bibliography, run
\begin{lstlisting}
lualatex main_thesis.tex
biber main_thesis
lualatex main_thesis.tex
lualatex main_thesis.tex
\end{lstlisting}
Of course, running these four commands manually every time is unnecessary. All \LaTeX editors I ever used (Sublime Text, VS Code, TeXstudio, TeXmaker, TeXShop, and Overleaf), allow you to set \verb|lualatex| as the default compiler and \verb|biber| as the bibliography processor. Additionally, all these editors enable you to define this sequence of commands as the standard build process, so your document is compiled correctly with a single action.

\paragraph{Why lualatex?} 
This template uses \verb|lualatex| instead of \verb|pdflatex| for two main reasons:
(i) Font Management: The template relies on the \verb|fontspec| package to handle different font types (e.g., sans-serif for captions and serif for the main text), which requires lualatex for compilation.
(ii) Memory Management: lualatex has a superior memory allocation system that dynamically adjusts as needed. This is particularly important when creating figures in \LaTeX with TikZ/pgfplots, where even simple 2D surface plots can lead to out-of-memory errors with \verb|pdflatex|.

Beyond these two reasons, \verb|lualatex| offers several other advantages, such as powerful Lua scripting capabilities. If you're interested, you can find numerous discussions on the benefits of \verb|lualatex| on \href{tex.stackexchange.com}{TeX Stack Exchange}.

\paragraph{Why biber and biblatex?}

This template uses \verb|biblatex| with the \verb|biber| backend instead of \verb|bibtex|. \verb|Biblatex| is more modern and versatile, and there’s virtually no reason to prefer \verb|bibtex| over \verb|biblatex|. The continued popularity of \verb|bibtex| is likely due to (bad) habit, both among users and some journals that still mandate its use.

For a thesis with a straightforward bibliography, \verb|bibtex| might be sufficient. However, \verb|biblatex| makes customization easier. Moreover, for more complex bibliography structures---such as our CRC183 project, which involves multiple independent bibliographies, cross-references between them, chapter-specific citation highlighting, and automatic correction of incorrect entries---\verb|bibtex| would have been completely inadequate.
See \chapref{chap:biblatex} for details.



\section{Documentclass}
This template is built on the \href{https://ctan.org/pkg/koma-script?lang=de}{KOMA-script} class \verb|scrbook|. Whether to use the standard \verb|book| class or its KOMA-Script counterpart is largely a matter of personal preference. The KOMA-Script bundle is widely recognized as an excellent package, and personally, I find \verb|scrbook| easier to customize while also preferring its default layout.

The document class is loaded with the following options:
\begin{lstlisting}
\documentclass[numbers=noenddot, headinclude = true,
	BCOR = 12mm, DIV = 16, twoside]{scrbook}
\end{lstlisting}
Below is an explanation of some of these options.

\subsection{Page layout without the \texttt{geometry} package}
One of the key strengths of the KOMA-Script classes is their built-in handling of type area and other typographic settings. The package author has extensive expertise in typography and has incorporated many best practices directly into its implementation. The KOMA-Script manual states:
 \say{Various algorithms and heuristic methods for constructing an appropriate type area have been discussed in the literature. These rules are known as the `canons of page construction.' [\ldots] The result is that the aspect ratio of the type area corresponds to the proportions of the page. [\ldots] In a two-sided document (e.g. a book), however, the entire inner margin (the margin at the spine) should be the same size as each of the two outer margins.} \cite{koma_manual}

The scrbook class automatically generates margins that conform to these typographic principles. The DIV parameter controls the type area width: the larger the value, the smaller the margins. The underlying construction mechanism is explained in Section 2.2 of the KOMA-Script manual, particularly in Figure 2.1.

Furthermore, \verb|headinclude=true| ensures that the header is included in the type area calculations as part of the text, and \verb|BCOR=12mm| adds a binding correction to the inner margin.

The KOMA-Script eliminates the need for the \verb|geometry| package that is otherwise often used to adjust margins. Manually setting margins can unintentionally violate typographic rules, potentially leading to poor readability. As the KOMA-Script manual warns:

\say{The practice of doing things oneself has long been widespread, but the results are often dubious because amateur typographers do not see what is wrong and cannot know what is important. This is how you get used to incorrect and poor typography. [...] Now, the objection could be made that typography is a matter of taste. When it comes to decoration, one could perhaps accept that argument, but since typography is primarily about information, not only can mistakes irritate, but they may even cause damage.}
(I recommend everyone to read Sec.~2.X of the manual, I enjoyed its polemic tone)

\paragraph{A confession} After emphasizing the importance of adhering to typographic best practices, I must confess: this template violates them.

For each combination of font size, font type (serif vs. sans-serif), and paper size, there is an optimal maximum number of characters per line to ensure readability. This determines the highest recommended \verb|DIV| value. The template uses \verb|DIV=16|, which exceeds the recommended value. As a result, LaTeX issues a \verb|Bad type area settings!| warning and suggests decreasing the \verb|DIV| value.
However, I deliberately chose smaller margins because I personally prefer this layout. My poor justification is that the recommended maximum character count per line is intended for inexperienced readers, whereas academic writing assumes a more advanced audience accustomed to dense text.



\section{Structure of the thesis}

According to the \href{https://mathnat.uni-koeln.de/sites/dekanat/official/Ordnungen/Promotionsordnung_2020.pdf}{`Promotionsordnung der math.-nat. Fakultät (2020)'} (PO2020), a Ph.D. thesis must include
\begin{itemize}
	\item a cover / title page,
	\item an abstract (PO2020 specifies that an English abstract is sufficient; a German version is not required),
	\item several chapters including an introduction, presentation of results, and discussion,
	\item the closing statement from §7 Absatz 8, PO2020, along with a list of your publications.
\end{itemize}
It is also common to include a thesis outline and acknowledgments. There is no universal consensus on where acknowledgments should be placed. In this template, they appear between the bibliography and the closing statement. However, other placements, such as directly after the title page and before the abstract, are also common. If I were to write my thesis again, I would likely choose that option.

Additionally, you should list your referees. In the published version (i.e., the version uploaded to the KUPS server after your defense, not the initial submission), you must also include the date of your defense and a statement confirming that the thesis was accepted by the faculty. This information should be printed either on the title page or its back. In this template, it is placed on the back of the title page.

A summary of the complete thesis structure is given in \tabref{tab:thesisstructure}.

\begin{table}
	\centering 
	\caption{\textbf{Structure of the thesis.}}
	\label{tab:thesistructure}
	\vspace{5ex}
	\begin{tabular}{lccccc} 
		\toprule
		 & page numbers & even/odd & in ToC & in pdf ToC & numbered  \\ 
		\midrule 
		Titlepage front  & none & odd & \color{bqred}\XSolidBrush& \color{bqgreen} \CheckmarkBold &\color{bqred} \XSolidBrush\\
		Titlepage back & none & even & \color{bqred}\XSolidBrush& \color{bqred}\XSolidBrush& \color{bqred}\XSolidBrush\\
		Abstract & none & odd & \color{bqred}\XSolidBrush& \color{bqgreen}\CheckmarkBold & \color{bqred}\XSolidBrush\\
		Table of Contents & roman & odd & \color{bqred}\XSolidBrush& \color{bqgreen}\CheckmarkBold &\color{bqred} \XSolidBrush\\
		Outline & arabic & odd &\color{bqgreen} \CheckmarkBold &\color{bqgreen} \CheckmarkBold & \color{bqred}\XSolidBrush\\
		Part page & none & odd & \color{bqgreen}\CheckmarkBold & \color{bqgreen}\CheckmarkBold & roman letters\\
		Chaptes & arabic & odd & \color{bqgreen}\CheckmarkBold &\color{bqgreen} \CheckmarkBold & numbers\\
		Appendices & arabic & odd & \color{bqgreen}\CheckmarkBold &\color{bqgreen} \CheckmarkBold & letters\\
		Bibliography & arabic & odd & \color{bqgreen}\CheckmarkBold & \color{bqgreen}\CheckmarkBold & \color{bqred}\XSolidBrush\\
		Acknowledgements & none & odd & \color{bqred}\XSolidBrush& \color{bqgreen}\CheckmarkBold & \color{bqred}\XSolidBrush\\
		Erklärung & none & odd & \color{bqred}\XSolidBrush&\color{bqgreen} \CheckmarkBold & \color{bqred}\XSolidBrush\\
		\bottomrule
	\end{tabular}
\end{table}

\subsection{Frontmatter, mainmatter and page numbering}
Books are typically divided into front matter and main matter (the back matter, which includes acknowledgments and the final statement, is not relevant for the following discussion since it uses the CHECLe
). In the \verb|tex| document, front matter and main matter are initialized via \verb|\frontmatter| and \verb|\mainmatter|. The front matter contains everything preceding the main content. It is common practice to use a different page numbering style for this section, typically Roman numerals or no numbering at all, while the main matter uses Arabic numerals.
In this thesis, the main matter begins with the Outline chapter. The front matter starts with the title page, but I opted not to display Roman numerals on the title page and abstract. Only the table of contents carries Roman numeral page numbers.

\subsection{Table of content}
The entire front matter should not be included in the table of contents (ToC). The same applies to the acknowledgments (which are often part of the front matter) and I also excluded the closing statement. However, I wanted all these sections to appear in the PDF table of contents (PDF-ToC), i.e., the strcuture that is displayed in the navigation pane of most PDF readers, see \figref{fig:pdf-toc}.

\begin{figure}
	\includegraphics[width = \textwidth]{pdf-toc.png}
	\caption{\textbf{Table of content vs. pdf table of content.} Thesis parts like the title page, abstract or acknowledgments should not be typeset in the table of content. However, for ease of navigation, it is useful if they appear in the navigation panel of the pdf reader.}
	\label{fig:pdf-toc}
\end{figure}

To customize the appearance of both the internal ToC and the PDF-ToC, the following commands are used:
\begin{itemize}
	\item \verb|\addcontentsline{toc}{chapter}{name}|  manually adds an entry to the ToC.  For example, \verb|\addcontentsline{toc}{chapter}{Bibliography}| ensures that the bibliography appears as a separate chapter in the ToC.
	\item \verb|\pdfbookmark[level]{Abstract}{abstract}| adds an entry to the PDF-ToC.
	\item \verb|\phantomsection| is sometimes necessary to ensure that hyperlinks between the PDF-ToC and the main document function correctly.
	\item \verb|\addchap{Outline}| ensures that the unnumbered Outline chapter appears both in the PDF-ToC and the ToC.
\end{itemize} 


\paragraph{Fixing the level in the pdf-toc}
In some cases, you may need to adjust the level at which an entry appears in the PDF-ToC. For example, consider the following thesis structure:
\begin{lstlisting}
\part{Part I}
	\chapter{Chapter 1}
	% ... 
\part{Part II}
	\chapter{Chapter 5}
	% ...
\chapter{Conclusion}
\end{lstlisting}
By default, the Conclusion chapter would be treated as a subchapter of Part II. However, you may prefer it to appear at the same level as the two parts. Here are two workarounds:
Before \verb|\chapter{Conclusion}|, insert
\begin{lstlisting}
\makeatletter
\def\toclevel@chapter{-1}
\makeatother
\chapter{Concluding discussion}
\end{lstlisting}
or load the bookmark package and insert
\begin{lstlisting}
\usepackage{bookmark}
...

\bookmarksetup{startatroot}
\chapter{Conclusion}
\end{lstlisting}


\section{Folder structure of the template}
I structured my thesis in the following way: The root folder contains the main file (\verb|main_thesis.tex|) and four subfolders, \verb|Latex|, \verb|Auxiliary|, \verb|bib-files|, \verb|chapters|. ADD FONT FILES / GITIGNORE.
\paragraph{bib-files} (unsurprisingly) contains all \verb|.bib|-files with the bibtex/biblatex entries.
\paragraph{chapters} contains a subfolder for each main chapter and the appendices, see below.
\paragraph{auxiliary} contains all the auxiliary parts of the thesis not contained in the \verb|chapters| folder, i.e., a separate tex file for titlepage, abstract, outline, acknowledgments and the final statement, as well as the folder \verb|logos| with the logos of the University of Cologne and the THP.
\paragraph{Latex} contains additional latex files with useful commands, TikZ style settings, color definitions, biblatex settings, etc.

Each subfolder in \verb|chapters| contains a main tex-file for that chapter as well as a \verb|figures| folder containing all figures for that chapter.
At the beginning of each chapter, it is useful to set the graphicspath to the respective figure folder, e.g., \verb|\graphicspath{{chapters/1_basicfeatures/figures/}}| for this chapter, such that one can ommit the graphics path when using the \verb|\includegraphics[]{}| command.

\begin{forest}
  pic dir tree,
  where level=0{}{% folder icons by default; override using file for file icons
    directory,
  },
  [root
    [main\_thesis.tex, file]
    [\fname{auxiliary}{all files for front and back matter}
      [\fname{abstract.tex}{one file for each front/back matter section: abstract, title page, acknowledgments, final statement.}, file]
      [\fname{logos}{Logos for title page}]
    ]
    [\fname{bib-files}{files with bibtex/biblatex entries}
    	[\fname{bibliography1.bib}{One or more .bib files}, file]
    ]
    [\fname{chapters}{main content}
    	[1\_introduction
    		[\fname{1\_introduction.tex}{main .tex file for chapter 1}, file]
    		[\fname{figures}{figures for chapter 1}]
    		[\fname{tikz\_figures}{TikZ source code for figures in chapter 1}
    		[\fname{tikz\_chap1\_fig1.tex}{A tex file for each figure in chapter 1}, file]
    		]
    	]
    	[\fname{...}{one folders for each chapter / appendix with equal substructure}]
    	[outline.tex, file]
    ]
    [\fname{latex}{folder for style definitions}
    	[\fname{layout.tex}{Layout settings}, file]
    	[\fname{...}{more files for TikZ styles, color definitions, custom \LaTeX commands, BibLaTeX settings, etc.}, file]
    ]
    [.gitignore, file]
    [\fname{otf/tff files with fonts}{optional font files}, file]
  ]
\end{forest}

\subsection{Compiling parts of the thesis}
All individual tex files after the \verb|\begin{document}| are imported using \verb|\include{filename}|. If you want to compile only parts of the thesis, put an \verb|\includeonly{filename1,filename,...}| in the preamble. Due to the separate \verb|.aux| files that \verb|\include| produces, one then obtains still the correct numbering of pages/chapters/etc. Anyway, this is probably not a big issue during the writing process and using \verb|\import{filename}| instead is of course also fine.

\section{Using TikZ / pgfplots and externalization}
When you use TikZ / pgfplots to generate your plots, you should use the externalization capabilities of TikZ in order to prevent re-rendering of unchanged figures whenever you compile. That saves a lot (!) of time. 
Remember that you have to include \verb|\usepackage{shellesc}| and compile with the \verb|--shell-escape| flag. Besides these general adjustements, there are two template-specific pecularities one has to keep in mind: 

\paragraph{Chapter headings} The modified chapter heading uses TikZ. Therefore, one should disable the externalization of TikZ before every new chapter. You can automate this by uncommenting this code snippet here in line ... of \verb|main_thesis.tex|:
\begin{lstlisting}
\let\oldchapter\chapter	% Store \chapter in \oldchapter
\renewcommand{\chapter}{%
	\tikzexternaldisable
	\oldchapter%
}
\end{lstlisting}
\paragraph{Modified folder structure}
I keep an additional folder \verb|tikz_figures| folder to store a separate .tex file with the tikz source code for every figure. That folder also contains the subfolder \verb|data| that in turn contains the data that is needed in the source code tikzfiles. At the beginning of each section, I set some paths to facilitate the inclusion of figures, for example
\begin{lstlisting}
\renewcommand{\here}{chapters/3_colors/tikz_figures}% Path to tikz source code
\pgfplotsset{table/search path={\here/data}}				% Path to data used in tikz source code
\tikzsetexternalprefix{chapters/3_colors/figures/}	% Path for storing of externalized tikz pdfs.
\end{lstlisting}
The \verb|\here| command is originally defined in the preamble of the document.
Then, the TikZ source code / the external pdf file is included via
\begin{lstlisting}
\begin{figure}
	\centering
	\tikzsetnextfilename{fig_colordemo}
	\input{\here/tikz_colordemo.tex}
	\caption{Some caption}
	\label{fig:colordemo}
\end{figure}
\end{lstlisting}
To see an example in action, checkout \chapref{cha:colors}.
Of course, there are many other options on how to organize the folder structure, in particular with regard to the externalization of TikZ-figures, just do what suits you best.

\section{``Outer theme'' layouting}

\subsection{Chapter headings}

\subsection{Part pages}

\subsection{Header}




\chapter{Colors}
\chapter{Bibliography with biblatex}\label{chap:biblatex}

\chapter{General advice}
use newcommands for ...
read the promotionsordnung
whatever you do, do it consistently
In case of trouble, first step is to delete cached files


In this first chapter, I briefly describe some general properties of the template: How to compile, what is the general structure, general remark about the page layouts, etc.

\section{Compilation, documentclass and page layout}

To compile the entire document, including the bibliography, run
\begin{lstlisting}
lualatex main_thesis.tex
biber main_thesis
lualatex main_thesis.tex
lualatex main_thesis.tex
\end{lstlisting}

This should result in a pdf-document where all references are correctly displayed and all cross-links between the bibliography in the main text works flawlessly.
Of course, it should not be necessary to run these four commands manually in the terminal. Every tex editor I ever used (SublimeText, VS Code, TeXstudio, TeXmaker, TeXlive) allows one to choose 
this succession of four commands as the standard build command that is executed whenever you compile your document.


\subsection{The document class}
This template uses the KOMA-script class \verb|scrbook|. There are many discussion on the pros and cons of KOMA-script classes over the regular \verb|book| class. I personally find the \verb|scrbook| easier to adjust to my wishes and slighly prefer the standard layout it produces. There is general consensus that the KOMA-script is a great bundle.

\paragraph{Page layout} One often sees people using the \verb|geometry| package to adjust the margins of the page. That is certainly a cool package, but in my opinion, don't use it, unless you really now what you are doing. There are surprisingly many things you can screw up, or stated otherwise, typographers have thought about good guidelines for layouting a page for years, so don't screw it up fahrlaessig.  
Here is a quotation from the KOMA-script manual:
``'Various algorithms and heuristic methods for constructing an appropriate type area have been
discussed in the literature. These rules are known as the ``canons of page construction.'' \ldots The result is that the
aspect ratio of the type area corresponds to the proportions of the page. In a one-sided document,
the left and right margins should have equal widths, while the ratio of the top and bottom margins
should be 1:2. In a two-sided document (e.g. a book), however, the entire inner margin (the margin at the spine) should be the same size as each of the two outer margins \ldots.'

The easiest way to make sure that the text area has the same ratio as the page is as follows:

\href{https://markov.htwsaar.de/tex-archive/macros/latex/contrib/koma-script/doc/scrguide-en.pdf}{How page layout is calculated.}

\begin{lstlisting}
\documentclass[BCOR = 12mm, DIV = 16, ...]{scrbook}
\end{lstlisting}


but that gives the Warning

I personnaly find that the margins produced with the default settings are too wide, therefore I use the large DIV value of 16 (I am sure that the KOMA-script author would consider this as a sacrileg). This gives me the following warning when I compile.

\begin{lstlisting}
Package typearea Warning: Bad type area settings! The detected line width is about 17\% larger than the heuristically estimated maximum limit of typographical good line width. You should e.g. decrease DIV, increase fontsize or change papersize.
\end{lstlisting}

But as I am quite happy with the final result, I just ignore these messages.

HOW T OADD LINEBREAK INS CAPTIONS!



\blindtext
\section{Here is a section}
\blindtext
\subsection{Here is a subsection}
\blindtext
\subsubsection{This is a subsubsection}
I never use subsubsections.
\paragraph{Paragraph} but sometimes I use paragraphs.
\section{How to cite papers.}
Blablabla, have a look at this cool paper \cite{berke_transmon_2022}. Note the `P' because it is one of \highlight{my} papers. Here is a regular paper \cite{aruteQuantumSupremacyUsing2019a} with a million authors, such that BibLaTeX uses (depending on your settings) \textit{et al.} and here is another paper \cite{magesan_effective_2020} with only two authors. Finally, here is one of my papers that is not yet published, and, therefore, I decided that it deserves its own category \cite{inpreparation}.

\section{How to make nice tables?}
Most importantly, do not use vertical rules!
From `The Chicago Manual of Style' \cite{chicagoMOS}: ``To produce a clear, professional-looking table, rules should be used sparingly. Many tables will require just three rules, all of them horizontal—one at the very top of the table, below the title and above the column heads; one just below the column heads; and one at the bottom of the table, along the bottom of the last row, above any notes to the table. (\ldots) Vertical rules should be used sparingly (\ldots).'' 
Use the \verb|booktabs| package, see  \href{https://ctan.org/pkg/booktabs}{here} for the documentation and \href{https://nhigham.com/2019/11/19/better-latex-tables-with-booktabs/}{here} or \href{}{} for some examples and a discussion why \verb|booktabs| is the way to go. Remember that table captions usually go above the table. See \tabref{tab:table1} for an example with multiple hierachy levels in $x$ and $y$ direction and \tabref{tab:table2} for an example of a side-by-side table with colored rows.

\begin{table}
	
	\centering
	\caption{\textbf{Example for table with different hierachy levels in $x$ and $y$ direction.} \blindtext}
	\label{tab:table1}
	\vspace{5ex}
	\begin{tabular}{@{}rrrrcrrr@{}}\toprule
		& \multicolumn{3}{c}{$w = 8$} & \phantom{abc}& \multicolumn{3}{c}{$w = 16$} \\
		\cmidrule{2-4} \cmidrule{6-8}
		& $t=0$ & $t=1$ & $t=2$ && $t=0$ & $t=1$ & $t=2$\\ 
		\midrule
		$\mathrm{dir}=1$\\
		$c$ & 0.0790 & 0.1692 & 0.2945 && 0.3670 & 0.7187 & 3.1815 \\
		$c$ & -0.8651& 50.0476& 5.9384&& -9.0714& 297.0923& 46.2143\\
		$c$ & 124.2756& -50.9612& -14.2721&& 128.2265& -630.5455& -381.0930\\
		$\mathrm{dir}=0$\\
		$c$ & 0.0357& 1.2473& 0.2119&& 0.3593& -0.2755& 2.1764\\
		$c$ & -17.9048& -37.1111& 8.8591&& -30.7381& -9.5952& -3.0000\\
		$c$ & 105.5518& 232.1160& -94.7351&& 100.2497& 141.2778& -259.7326\\
		\bottomrule
	\end{tabular}

\end{table}

\begin{table}
	\centering 
	\caption{\textbf{Example of side-by-side table and colored rows.} \blindtext}
	\label{tab:table2}
	\vspace{5ex}
	\begin{tabular}{ccccrr} 
		\toprule
		$l_1$ & $l_2$ & $l_3$ & $l_4$ & $\Delta N_\text{ex}$ & $|\langle \psi | \op{H}_\text{int} | \phi \rangle|$  \\ 
		\midrule 
		\rowcolor{pqred} 0 & 1 & 3 & 0 & 0 & ---\\
		0 & 0 & 0 & 0 & $-4$ & 0.04\\
		0 & 0 & 2 & 0 & $-2$ & 1.88\\
		\rowcolor{pqblue} 0 & 0 & 4 & 0 & 0 & 2.06\\
		0 & 0 & 6 & 0 & 2 & 0.32\\
		0 & 0 & 8 & 0 & 4 & 0.09\\
		0 & 1 & 0 & 1 & $-2$ & 0.04\\
		\rowcolor{pqyellow} 0 & 1 & 0 & 3 & 0 & 0.002\\
		0 & 1 & 0 & 5 & 2 & 0.0001\\
		0 & 1 & 0 & 7 & 4 & 0.00002\\
		\rowcolor{pqblue} 0 & 1 & 2 & 1 & 0 & 1.89\\
		\bottomrule
	\end{tabular}
	\hspace{0.5cm}
	\begin{tabular}{ccccrr} 
		\toprule
		$l_1$ & $l_2$ & $l_3$ & $l_4$ & $\Delta N_\text{ex}$ & $|\langle \psi | \op{H}_\text{int} | \phi \rangle|$  \\ 
		\midrule 
		0 & 2 & 0 & 0 & $-2$ & 0.06\\
		\rowcolor{pqblue} 0 & 2 & 2 & 0 & 0 & 2.57\\
		0 & 2 & 4 & 0 & 2 & 2.82\\
		0 & 2 & 6 & 0 & 4 & 0.44\\
		\rowcolor{pqyellow} 0 & 4 & 0 & 0 & 0 & 0.003\\
		0 & 4 & 2 & 0 & 2 & 0.15\\
		0 & 4 & 4 & 0 & 4 & 0.17\\
		0 & 6 & 0 & 0 & 2 & 0.0005\\
		0 & 6 & 2 & 0 & 4 & 0.02\\
		0 & 8 & 0 & 0 & 4 & 0.0001\\
		\rowcolor{pqblue} 1 & 0 & 3 & 0 & 0 & 1.17\\
		\bottomrule
	\end{tabular}
\end{table}


\chapter{Notes on the use of colors}

Selecting well-suited colors is a surprisingly complex challenge. Besides the requirement to be aesthetically pleasing, the color schemes should at best
\begin{itemize}
	\item be distinct for color-blind readers,
	\item work in monochrome print out,
	\item work on screen and paper,
	\item respect `semantic resonances' \cite{linSelectingSemanticallyResonantColors2013} (e.g., `blue=cold',`red=hot' \footnote{Note that such associations can vary depending on cultural conditioning. The association of `blue' with `cold' is nearly universal, but, e.g., the colour of mourning is white in Japan, but black in many western cultures.}),
	\item be printer-friendly (`RGB vs. CMYK' issue).
\end{itemize}
The first point is probably the most important, considering that, for example, 6\% of all males have deutan color vision deficiency (`green-blindness'). Fortunately, there are many very good color picking tools, or predesigned color schemes tailored to colorblind people.
The references and tools that I have used most often are \href{https://personal.sron.nl/~pault/#sec:greyscale_conversion}{`Paul Tol's Notes'} and \href{https://colorbrewer2.org/#type=sequential&scheme=BuGn&n=3}{ColorBrewer}. 
The color palettes presented there have made it to a certain reputation. In `Paul Tol's notes', you will also find some details worth reading about the different types of color blindness and greyscale conversion.
Finally, \href{https://www.vis4.net/palettes/}{vis4.net} is a advanced and powerful tool based on \href{https://github.com/gka/chroma.js}{chroma.js} that helps you designing your own color palette, and it even shows you the perception of the chosen colors with the most frequent forms of color blindness. However, I was usually completely satisfied with the first two references.

In my thesis, I had four different main use cases for colors, some with specific sub-cases:
\begin{itemize}
	\item Line plots:
		\begin{itemize}
			\item qualitative,
			\item pairwise,
			\item sequential. 
		\end{itemize}
	\item surface plots:
		\begin{itemize}
			\item qualitative,
			\item divergent,
			\item sequential. 
		\end{itemize}
	\item colored text or text on colored background, e.g. filled cells in tables,
	\item other design elements or drawings.
\end{itemize}
For the last point (e.g. heading colors, link colors, or sketch of, e.g., a physical system like a pendulum, i.e., cases where there is no loss of information to fear if colors are perceived incorrectly) I have taken as the only criterion my personal taste. The other points are explained in more detail below.


\section{Define your own commands}
\verb|\figref|

\section{Example of figures and side-by-side figures.}

\section{Some stylistic advices}
Below is a collection of stilistic questions I have thought about for too long.
\subsection{Hyphen vs. en dash vs. em dash}
There a three lengths of hyphens, the regular hyphen \verb|-|, the en dash \verb|--| and the em dash \verb|---|.
This is what they look like: - vs. -- vs. ---.
In accordance with the Chicago manual of style, I used the hyphen only to connect words that function together as a single concept or work together as a joint modifier, i.e., `well-defined concept' or `collision-free device'. Never use the single hyphen similar to brackets.
The main application for the en dash is to connect things that are related by some form of distance, e.g. `pages 12--17', the `May--September issue of journal \ldots', the cycling race Milan--San Remo `or `a coupling strength of 10--15 MHz' (although in the last example I would rather write `10 to 15 MHz'. 
A use case I never encountered: an en dash is used in place of a hyphen in a compound adjective. Specifically, an en dash is preferred when one element of the compound is itself an open compound. For example, the prefix post- is usually connected to the following word with a hyphen, but to connect it to the compound noun World War II, it’s better to use an en dash, i.e., post--World War II vs. post-World War II.

The em dash should be used to separate additional thoughts---like this. Here is an example from my thesis: `Here, we argue that the insights gained for the simpler model---in particular, the existence of a quantum chaotic region for too low disorder---retain their validity for more elaborate frequency arrangements and geometries.' I use the em dashs without spaces as is usually recommended however some style guides --- not sure which one --- recommend the use of spaces (some times -- as here -- one encounters the combination of en dashes with spaces.

As always, the most important thing is consistency. Once you have decided on a variant, you should use it consistently everywhere and always.

\subsection{Nonlinear vs. non-linear and related issues}
The prefix `non' appears very often and with many different following words. My impression was that it is not treated uniformly in the literature (even by the same authors) and that one finds both variants. In fact, in American English (AE), the hyphen is mostly omitted, whereas in British English (BE) it is commonly used. Compare e.g., the two reviews \cite{abaninRecentProgressManybody2017} and \cite{abaninColloquiumManybodyLocalization2019}. Ref.~\cite{abaninRecentProgressManybody2017} was published by Annals of Physics (Berlin) which belongs to the german Wiley-VCH publisher and uses BE. Therefore one finds \emph{non-quilibrium,non-entangled,non-uniform,non-thermal,non-ergodic,non-local,non-trivial,non-interacting,non-Abelian,} etc.
Ref.~\cite{abaninColloquiumManybodyLocalization2019} was published in Reviews of modern physics, an APS journal. In consequence, they write \emph{nonequilibrium,nonzero,noninteracting,nonuniform,nonthermal,nonvanishing,nonlocal,nonthermalizing}, but however, they write \emph{non-Abelian}. I went for the AE option, because I used AE everywhere else (behavior, neighbor, gray, etc.).
When browsing papers to find out how issues like this are handled, the best idea is not to take the arXiv versions, as there is often no consistency even within a single paper, but the published versions that have undergone some form of post-editing by the publisher.


