\renewcommand{\here}{partII_transmons/Auxiliary/}
\addchap{Outline}
This thesis applies concepts from the theory of \highlight{many-body localization} (MBL) to the \highlight{transmon} platform for quantum computing to scrutinize prevalent processor architectures for the emergence of \highlight{quantum chaotic} behavior. In detail, this work is organized as follows: 

\textbf{Chapter 1} intends to give readers without prior knowledge of quantum computing a brief outline of the subject. It positions the transmon in the zoo of possible qubit platforms and discusses its leading role in the current NISQ era. 
We introduce some frequently used basic vocabulary. Following this, \textbf{Chapter 2} acquaints with the concepts of many-body localization and quantum chaos. The discussion highlights which properties of the chaotic phase constitute a fundamental obstacle for quantum computing. 
% With this, we attempt to circumvent a weakness of many introductions to quantum chaos, which often equate chaos with level repulsion. Why such systems cannot process information remains in the dark.
Furthermore, a selection of diagnostic tools is introduced, based on which one can distinguish between the chaotic and the harmless regime.
In \textbf{Chapter 3}, attention is focused exclusively on this thesis's main protagonist: the superconducting transmon qubit. We discuss its properties in detail and show how many transmons combined can form a quantum computer.
Arrays of interconnected transmons are the main object of interest to which we apply the MBL toolbox.
Contemporary processors from companies like IBM or Google serve as the blueprint for our simulations. Therefore,
the second half of the chapter takes a closer look at the design philosophies to which different groups subscribe. In particular, this discussion emphasizes the close link between the choice of a specific gate implementation and the parameters of the static `gate-off' Hamiltonian.
The insights from this analysis guide the parameter choices for the simulations.
\textbf{Chapter 4} then provides a detailed and rigorous discussion of the transmon processor from the many-body perspective. We show that a regime of dangerous chaotic fluctuations can be found in some pervasive design classes. A particular focus lies on establishing quality indicators for parameter regimes that show deviations from deep localization but are not in the immediate vicinity of hard quantum chaos.
The focus in \textbf{Chapter 5} is on the critical examination of one of the most recent engineering enhancements---the implementation of frequency patterns to augment the precision in the entangling gate operations---from the many-body perspective.  
For \textbf{Chapter 6}, we switch perspectives and solve the classical equations of motion for a system of coupled transmons. In the classical limit, a transmon reduces to a mathematical pendulum. Our quantitative analysis of these nonlinear pendulum systems reveals a striking similarity between the quantum mechanical and the classical system regarding their susceptibility to chaos. We conjecture that classical analysis can be a tool to gauge the quality of specific parameter configurations for system sizes beyond the feasibility of a quantum mechanical simulation.
Finally, \textbf{Chapter 7} combines a broad summary, concluding remarks, and a brief outlook on future research directions. In particular, we highlight again why merging the theoretical field of many-body physics with the applied field of quantum chip design is fruitful and why quantum engineers should include many-body concepts in their efforts to design viable processor architectures.

These chapters are further supplemented by three appendices, providing examples of state-of-the-art transmon processors, additional results, and an extended discussion of some of the more technical aspects of this work, including details on the implementation.

Last mentioned, \appref{app:kitaev} provides a concise overview of results obtained in an independent branch of research (field-induced effects in Kitaev spin liquids) conducted during the course of this thesis.