% Abstract
\pdfbookmark{Abstract}{abstract}
\cleardoublepage
\thispagestyle{empty}

\begin{center}
{\color{maincolor2} \mainregular \Huge Abstract} 
\end{center}
\vspace{1cm}

\noindent The quest for quantum computers is in full swing. Over the past decade, the frontiers of quantum computing have broadened from exploring few-qubit devices to developing viable multi-qubit processors. One of the protagonists of the present era is the superconducting transmon qubit. By harmoniously combining applied engineering with fundamental research in computer science and physics, transmon-based quantum processors have matured to a remarkable level. Their applications include the study of topological and nonequilibrium states of matter, and it is argued that they have already ushered us into the era of quantum advantage. Nevertheless, building a quantum computer that can solve problems of practical relevance remains a massive challenge. As the field progresses with unbridled panache, the question of whether we have a comprehensive picture of the potential dangers lurking in the wings acquires increasing urgency. In particular, it needs to be thoroughly clarified whether, with viable quantum computers of $\mathcal{O}(50)$ qubits at hand, new and hitherto unconsidered obstacles associated with the \highlight{multi}-qubit nature can emerge.
For example, the high accuracy of quantum gates in small-scale devices is hard to obtain in larger processors. On the hardware side, the unique requirements posed by large quantum computers have already spawned new approaches to qubit design, control, and readout.

This thesis introduces a novel, less applied perspective on multi-qubit processors.
Specifically, we fuse the field of quantum engineering and many-body physics by applying concepts from the theories of localization and quantum chaos to multi-transmon arrays. From a many-body perspective, transmon architectures are synthetic systems of interacting and disordered nonlinear quantum oscillators. While a certain amount of coupling between the transmons is indispensable for performing elementary gate operations, a delicate balancing with disorder---site-to-site variations in the qubit frequencies---is required to prevent locally injected information from dispersing in extended many-body states. 
Transmon research has established different modalities to cope with this dilemma between inefficiency (slow gates due to small coupling or large disorder) and information loss (large couplings or too small disorder).
We analyze small instances of transmon quantum computers in exact diagonalization studies, using contemporary quantum processors as blueprints.
Scrutinizing the spectrum, many-body wave functions, and qubit-qubit correlations for experimentally relevant parameter regimes reveals that some of the prevalent transmon design schemes operate close to a region of uncontrollable chaotic fluctuations. 
Furthermore, we establish a close link between the advent of chaos in the classical limit and the emergence of quantum chaotic signatures. 
Our concepts complement the traditional few-qubit picture that is commonly exploited to optimize device configurations on small scales. 
Destabilizing mechanisms beyond this local scale can be detected from our fresh perspective. This suggests that techniques developed in the field of many-body localization should become an integral part of future transmon processor engineering.





